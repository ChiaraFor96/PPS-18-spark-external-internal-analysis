\documentclass[12pt,italian]{article}
\usepackage[margin=1in]{geometry}
\setlength{\parskip}{5pt}
\usepackage[utf8]{inputenc} 
\usepackage[italian]{babel}
\usepackage{hyperref}
\usepackage{float}
\usepackage{times}

\title{PPS18  - ``spark-external-internal-analysis''}
\author{Chiara Forresi, matr: 880050, email: {\url{chiara.forresi@studio.unibo.it}}}
\date{\today}

\begin{document}

\maketitle
\newpage
\tableofcontents
\newpage

\section{Cos'è Spark?}
Spark non è solo un motore di elaborazione di big data, può essere considerato un ``ecosistema" che offre svariate possibilità.
%TODO
\par Vista la varietà di funzionalità che Spark offre, in questo progetto ci si focalizzerà maggiormente sulle funzionalità basilari di SparkCore e sulle modalità di gestione di Streaming di dati. In entrambi i casi l'analisi verrà svolta si da un punto di vista estereriore di quello che offre il framework che quello interiore, andando a sviscerare le caretteristiche e i punti peculiari (positivi o negativi) che emergono dall'analisi. 
\newline
Si sottolinea che in questo progetto è stata utilizzata l'ultima versione disponibile di Spark, ovvero la \textbf{2.4.4}. Per cui per maggiori dettagli si rimanda al sito di \href{https://spark.apache.org/docs/2.4.4/quick-start.html}{Spark}.

\section{Moduli per lavorare con stream di dati}
\subsection{Spark Streaming}
\subsection{Spark Structured Streaming}

%\begin{figure}[H]
%	\centering 
%	\includegraphics[width=0.6\linewidth]{img/SmartPositioner.png}
%	\caption{}
%	\label{fig:ClassDiagram}
%\end{figure}

\end{document}
    
    
